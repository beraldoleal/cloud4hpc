
\documentclass[11pt, titlepage]{article}
\usepackage[top=20mm, bottom=20mm, left=20mm, right=20mm]{geometry}
\geometry{a4paper}
\geometry{portrait}
\usepackage[utf8]{inputenc}	% Comment out for xelatex; must use with pdflatex
\usepackage[T1]{fontenc}
\usepackage[parfill]{parskip} % Activate to begin paragraphs with an empty line rather than an indent
\usepackage{graphicx}
\usepackage{amsmath}
\usepackage{amssymb}
%\usepackage{fontspec}
\usepackage{epstopdf}
\usepackage{float}
\restylefloat{table}
\usepackage{xcolor}
\usepackage{colortbl}
%\usepackage{booktabs}
\usepackage{tabu}
\usepackage{caption}
\usepackage{subcaption}
\captionsetup[figure]{labelfont=sc}
\captionsetup[table]{labelfont=sc}
\captionsetup[tabu]{labelfont=sc}

\DeclareCaptionType{MPEquation}[][List of equations]
\captionsetup[MPEquation]{labelformat=empty}

\begin{document}
\title{Otimizando a Localização de MVs na Nuvem Através de Padrões de Comunicação}\author{Artur Baruchi}
\maketitle
\section{Abstract}
\label{MPSection:916E337A-8E01-4963-AF49-BE8C7DC6C2A9}

O uso de computação na nuvem por usuários de Computação de Alto Desmpenho (HPC -
      \emph{High Performance Computing}) ainda sofre muita resistência, em grande medida, por conta da grande degradação de desempenho que essas aplicações sofrem ao serem executadas na nuvem. Entre os principais fatores que levam à degradação, pode-se citar (1) interconexão de rede lenta, que geralmente faz uso de hardaware commodity, (2) ambiente heterôgeneo com diferentes gerações de processadores, (3) compartilhamento do hardware entre diferentes usuários e (4) sobrecarga da virtualização.

Diferentes estratégias para viabilizar HPC na nuvem vem sendo discutidas e propostas. Em geral, pode-se classificar essas estratégias em Cloud-Aware, quando a aplicação HPC tem ciência de que está sendo executada em um ambiente na nuvem e de suas limitações ou HPC-Aware, que são estratégias que visam configurar a nuvem de forma otimizada para a execução de aplicações HPC. Neste trabalho, será apresentada uma solução que orbita as estratégias HPC-Aware, em que baseando-se no padrão de comunicação da aplicação, as Máquinas Virtuais (VM - Virtual Machine) serão realocadas em hospedeiros físicos com o objetivo de reduzir a sobrecarga inerente de um ambiente na nuvem para aplicações HPC. Nossa hipótese é que, dado um determinado padrão de comunicação poderiamos deixar MVs que trocam mensagens constantemente mais próximas (no mesmo hospedeiro ou no máximo a um hop de distância). Uma das possibilidades que também poderá ser avaliada é a alocação de MVs com padrões de comunicação incompatíveis entre si e separar MVs que estejam nessa situação.

\section{Keywords}
\label{MPSection:3D7588EF-0A74-4F02-AD69-655B1031BE95}

HPC, Cloud, Communication, Message Passing, Characterization

\section{Introdução}
\label{MPSection:44EB7A3A-0A0A-4C54-8447-1F6D5347559F}

A adesão de usuários de aplicações de Computação de Alto Desempenho (HPC - \emph{High Performance Computing}), seja no ambito acadêmico ou corporativo não é comum. Isso ocorre devido a diversos fatores, como interconexão de rede lenta, ambiente computacional heterogêneo, sobrecarga da virtualização e compartilhamento de recursos  . De fato, a computação na nuvem tem como principal argumento a redução de custos de manter um parque computacional complexo e, de forma geral, centros computacionais que tem a sua disposição supercomputadores são menos suscetíveis ao argumento do custo. Entretanto, usuários de aplicações HPC em centros computacionais de pequeno e médio porte (ex. Startups, pequenos laboratórios de pesquisa, etc) possuem maior sensibilidade a apelo da computação da nuvem e por essa razão mais suscetiveis a adotá-la.

Apesar do custo da computação na nuvem ser um atrativo, sua adoção ainda sofre grande resistência por parte de usuários de pequeno e médio porte de aplicações HPC em razão da degradação do desempenho. Em grande medida, a degradação de desempenho de aplicações HPC na nuvem deve-se ao uso de compartilhado de recursos computacionais, principalmente de interfaces de rede.



\section{Bibliograp}
\label{MPSection:EFC085AC-3D7A-48DF-E988-4F7FD8E6C8DB}

\section{Revisão Bibliográfica}
\label{MPSection:FA9A33D4-490F-406D-84E8-2BDA22A0F52D}



\section{Metodologia}
\label{MPSection:E3D67007-0EDE-487B-B714-6F004A58F4A3}



\section{Ambiente de Testes}
\label{MPSection:640E517E-5F30-41FD-9883-196E5C5C4E6E}





\section{}
\label{MPSection:681BA355-A6E6-49C3-CA78-2C39615C42BA}


\end{document}
